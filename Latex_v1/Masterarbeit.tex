 \documentclass{scrreprt}
\usepackage[utf8]{inputenc}
\usepackage{blindtext, xcolor}
\usepackage{comment}
\usepackage{enumerate}
\usepackage{multirow}
\usepackage[shortlabels]{enumitem}
\usepackage{graphicx}
\usepackage{mathtools}
\usepackage{float}
\usepackage{svg}
%\usepackage{glossaries}
\usepackage[acronym,numberedsection]{glossaries}
\usepackage{listings}
\usepackage{color}
% deutsche Silbentrennung
%\usepackage[ngerman]{babel}
% fuer Stichwortverzeichnis
\usepackage{makeidx}
\usepackage{marvosym}
\usepackage{scrpage2}
\usepackage{url}
\usepackage{setspace}
\onehalfspacing

\renewcommand*{\chapterheadstartvskip}{\vspace*{0cm}}
% http://tex.stackexchange.com/questions/19738/why-doesnt-pagestyleempty-work-on-the-first-page-of-a-chapter
\renewcommand*\chapterpagestyle{scrheadings}

%\restylefloat{table}

\definecolor{dkgreen}{rgb}{0,0.6,0}
\definecolor{mauve}{rgb}{0.58,0,0.82}
\definecolor{gray}{rgb}{0.4,0.4,0.4}
\definecolor{darkblue}{rgb}{0.0,0.0,0.6}
\definecolor{cyan}{rgb}{0.0,0.6,0.6}

\lstset{frame=tb,
	captionpos=b,
	language=Java,
	aboveskip=3mm,
	belowskip=3mm,
	showstringspaces=false,
	columns=flexible,
	basicstyle={\small\ttfamily},
	numbers=none,
	numberstyle=\tiny\color{gray},
	keywordstyle=\color{blue},
	commentstyle=\color{dkgreen},
	stringstyle=\color{mauve},
	breaklines=true,
	breakatwhitespace=true,
	tabsize=3
}

\lstdefinelanguage{XML}
{
	morestring=[b]",
	morestring=[s]{>}{<},
	morecomment=[s]{<?}{?>},
	stringstyle=\color{black},
	identifierstyle=\color{darkblue},
	keywordstyle=\color{cyan},
	morekeywords={xmlns,version,type}% list your attributes here
}

% Umbenennung einer Einträge
\renewcommand\tablename{Tabelle}
\renewcommand{\figurename}{Abbildung}
%With babel (and English as language):
%\addto\captionsenglish{\renewcommand{\figurename}{Fig.}}
\renewcommand\bibname{\section{Literaturverzeichnis}}
\renewcommand*\contentsname{Inhaltsverzeichnis}
%\renewcommand{\glossaryname}{Sachwortverzeichnis}
\renewcommand{\indexname}{\section{Sachwortverzeichnis}}
\renewcommand{\acronymname}{\section{Abkürzungsverzeichnis}}
\renewcommand{\listfigurename}{\section{Abbildungsverzeichnis}}
\renewcommand{\listtablename}{\section{Tabellenverzeichnis}}
\renewcommand\lstlistingname{Listing}
\renewcommand\lstlistlistingname{\section{Listingverzeichnis}}

% Damit auch subsubsections nummeriert werden und im ToC auftauchen
\setcounter{secnumdepth}{3}
\setcounter{tocdepth}{2}

\makeglossaries
\makeindex
\input{Grundstruktur/Abkuerzungen}
\input{Grundstruktur/Sachwoerter}
\pagestyle{scrheadings}
\clearscrheadfoot
\ihead{\includegraphics[height=1.69cm]{images/FH-burgenland-logo_white.png}}
\ohead{
	\\ \vspace{15px} \textnormal{Department Informationstechnologie und Informationsmanagement}
	%\par\nobreak\vspace{-8px}\makebox{\rule{\textwidth}{0.4pt}}
	\par\nobreak\vspace{-12px}\line(30,0){325}
	}
\ofoot{
	\pagemark
	\par\nobreak\vspace{-12px}\makebox[\linewidth]{\rule{\textwidth}{0.4pt}}
	\\
	\textnormal{Masterstudiengang Business Process Engineering \& Management}
	%\vspace{-60px}
}
%\setheadtopline{1pt}
%\setheadsepline{0.4pt}
%\setfootsepline{0.4pt}
%\setfootbotline{1pt}
\setlength{\headsep}{1.0in}
%\input{kopf-und-fusszeile2}
\begin{document}
% FOLGENDE ZEILE KANN EINEN FEHLER PRODUZIEREN!
% Die Zeile soll bewirken, dass auf der Titelseite keine
% Seitennummer angegeben wird. Falls das so nicht funktioniert,
% muss ein anderer Workaround herhalten oder die Fußziele in
% zwei Varianten angelegt werden.
% Die folgende Zeile muss auskommentiert werden, damit auf der Titelseite KEINE Seitenzahl angezeigt wird.
\begin{titlepage}
	\thispagestyle{scrheadings}
% Die folgende Zeile wurde eingefügt, um die Seitenzahl auf der Titelseite auszublenden ;
\thispagestyle{empty}
\vspace*{1cm}
%\begin{flushright}
%    \includegraphics{images/FH-burgenland-logo.png}
%\end{flushright}
\noindent
\vspace*{0.15cm}\text{Fachhochschule Burgenland GmbH}
\\
\vspace*{0.15cm}\text{Campus 1}
\\
\text{A-7000 Eisenstadt}
\vspace{3cm}

\begin{center}  % Diplomarbeit ODER Magisterarbeit ODER Dissertation
    \Huge{\textbf{\textsf{
         \textless\textless Titel der Arbeit\textgreater\textgreater
    }}}
    \vspace{1cm}

    \Large{\textsf{\textbf{
        Masterarbeit
        \\zur Erlangung des akademischen Grades
        \\Master of Science in Engineering (MSc)
    }}}
\end{center}
\vspace{1cm}

\vfill

\noindent\begin{tabular}{@{}ll}
\textsf{Betreuer:}
&
\textsf{\textless\textless Titel Vorname Familienname\textgreater\textgreater}
\\
\textsf{Eingereicht von:}
&
\textsf{\textless\textless Titel Vorname Familienname\textgreater\textgreater}
\\
\textsf{Personenkennzeichen:}
&
\textsf{\textless\textless Personenkennzeichen bitte ergänzen\textgreater\textgreater}
\\
\textsf{Datum:}
&
\textsf{01. Jänner 2015}
\end{tabular}
\end{titlepage}
\pagenumbering{Roman}
\setcounter{page}{1}
\newpage
\addcontentsline{toc}{chapter}{Vorwort}
\chapter*{Vorwort}
Hier ist der Platz für persönliche Worte wie zum Beispiel Dankesworte an Betreuer, Berater, Freunde und Familie.
\vspace{8cm}
\\
\begin{flushright}
	Franziska Musterfrau
	\\Eisenstadt, 01. Februar 2015
\end{flushright}
\newpage
\addcontentsline{toc}{chapter}{Kurzfassung}
\chapter*{Kurzfassung}
Hier ist der Inhalt der Arbeit in komprimierter Form darzustellen. 
\\Maximale Anzahl der Zeichen: 2.000 – 3.000
\\
\\
Anmerkung: 
Der Leser der Kurzfassung soll verstehen, welche Problemstellung / Fragestellung durch die vorliegende Arbeit bearbeitet wird und welche Erkenntnisse und Ergebnisse vorliegen.
Bitte beachten Sie auch die Richtlinie des Studiengangs zur Erstellung von wissenschaftlichen Arbeiten, vor allem die formalen Anforderungen wie z.B. Zitierweise nach APA-Styleguide in der Form (Autor, Jahr, Seite). 
\\
\\
\blindtext[1]
\newpage
\section*{Abstract}
Komprimierter Inhalt der Arbeit (Kurzfassung) in Englisch
\\Maximale Anzahl der Zeichen: 2.000 – 3.000
\\
\\
\blindtext[1]
\tableofcontents
\clearpage
\pagenumbering{arabic}
\setcounter{page}{1}
\chapter{Einleitung\cite{LoremIpsum}}
\blindtext[1]

\section{Problemstellung}
In der Problemstellung erfolgt eine Hinführung zum Thema aus dem globalen Zusammenhang heraus betrachtet: Warum es wichtig ist, sich mit dem konkreten Thema zu beschäftigen bzw. welche Bedeutung hat dieses Thema zum Beispiel für die Wirtschaft, Gesellschaft und Umwelt.

\section{Zielsetzung}
Hier werden die konkrete(n)wissenschaftliche(n) Fragestellung(en) klar angeführt. Neben dem inhaltlichen Ziel der Arbeit wird fallweise auch angegeben, wie vorgegangen wird, um die Fragestellung zu bearbeiten (z.B. mit einem Online-Fragebogen). Auch die Zielgruppe der Arbeit und der für die Zielgruppe angestrebte Nutzen kann an dieser Stelle angeführt werden.
\\
\\
Anmerkung: 
\\Die Erfahrung zeigt, dass es hilfreich ist, das erste Einleitungskapitel an die Betreuerin / an den Betreuer zu schicken und Feedback dazu zu erhalten.
\\Bitte auch darauf achten, dass alle Absatz-Abstände gleich groß sind.
\\Jedes Kapitel sollte mit einem Satz beginnen und einem Satz enden und ca. mind. 1/2 A4-Seite umfassen. 

\chapter{Grundlagen}

\section{Allgemeine Definitionen}
Die in dieser Formatvorlage beispielhaft enthaltenen Überschriften sind auf die im konkreten Fall tatsächlich passenden Überschriften anzupassen.
In diesem Teil der Arbeit werden die zum eindeutigen Verständnis unbedingt erforderlichen Grundlagen und Definitionen sowie die Erklärung wichtiger Begriffe angeführt.
Die Gliederungspunkte müssen möglichst prägnant bezeichnet werden.
\section{Stand des Wissens}
Auch die neuesten Entwicklungen und Arbeiten auf diesem Gebiet (Stand der Wissenschaft oder auch state-of-the-art) sind darzulegen, wobei diese je nach Thema auch in der 1. Gliederungsebene behandelt werden können.

\chapter{Vorgangsweise und Methoden}
\section{Vorgangsweise}
Alle im durchgeführten Untersuchungen und Versuche müssen systematisch und nachvollziehbar sein. Daher ist die gewählte Vorgangsweise genau zu beschreiben und zu begründen. Es empfiehlt sich, dafür Literatur zum wissenschaftlichen Arbeiten heranzuziehen.
\section{Methoden}
Die eingesetzte Methoden (z.B. Online-Befragung, Inhaltsanalyse, Interviews) müssen ebenfalls nachvollziehbar beschrieben werden. 
Unterschiedliche Untersuchungsmethoden haben oft unterschiedliche Genauigkeit.
Neben der Begründung und Beschreibung der Untersuchungsmethoden ist auch eine Begründung und Beschreibung der verwendeten Auswertungsmethoden bzw. dafür verwendete Software unerlässlich.
\subsection{\textless\textless Überschrift 3. Ebene\textgreater\textgreater}
\subsubsection{\textless\textless Überschrift 4. Ebene\textgreater\textgreater}
4 Überschriftenebenen müssen reichen.
\subsection{\textless\textless Überschrift 3. Ebene\textgreater\textgreater}
Wenn es ein Kapitel 3.2.1 gibt, muss es auch ein Kapitel 3.2.2 geben.

\chapter{Empirischer Teil}
Die Durchführung der empirischen Untersuchung ist nachvollziehbar zu dokumentieren sowie auch die dabei aufgetretenen Probleme und deren Behandlung. Der Umfang ergibt sich aus der Art der Bearbeitung. Tabelle ~\ref{table:HelloWorld} zeigt ein Bespiel für eine Tabelle. Abbildung ~\ref{fig:fhLogo} zeigt ein Beispiel für eine Abbildung.
\\
\\Nach dem Motte \acrfull{dry} wird dieses Akronym nur einmal definiert.
\begin{figure}[H]
	\centering
	\includegraphics[height=100px]{images/FH-burgenland-logo.png}
	\caption{FH-Logo}
	\label{fig:fhLogo}
\end{figure}
\begin{table}[H]
	\centering
\begin{tabular}{|l|l|}
	\hline
	\textbf{Spalte 1} & \textbf{Spalte 2} \\ \hline
	Hello             & World             \\ \hline
\end{tabular}
	\caption{HelloWorld}\label{table:HelloWorld}
\end{table}
\begin{table}[H]
	\centering
\begin{tabular}{|l|l|}
	\hline
	\textbf{Spalte 1} & \textbf{Spalte 2} \\ \hline
	Hello             & World             \\ \hline
\end{tabular}
	\caption{HelloWorld}\label{table:HelloWorld}
\end{table}
\chapter{Ergebnisse und Schlussfolgerungen}
Die Ergebnisse der Arbeit sind in übersichtlicher Form darzustellen Die gewählte Form der Darstellung ist vom gewählten Datenmaterial und den in der Einleitung gesetzten Zielen abhängig. Die Ergebnisse sind zu interpretieren und in Bezug zum Stand des Wissens zu diskutieren. Über die Beantwortung der Forschungsfrage und die daraus gezogenen Schlussfolgerungen schließt sich der Bogen zur Einleitung.
\\
\\Wichtig ist die gedanklich klare Unterscheidung zwischen der Darstellung der Ergebnisse und der Interpretation/Bewertung der Ergebnisse.

%	\begin{comment}

\chapter{Zusammenfassung und Ausblick}
\blindtext[3]
%\chapter{Verzeichnisse}
\cleardoublepage
%\appendix
%	\end{comment}
\chapter{Verzeichnisse}
%http://stackoverflow.com/questions/1243342/how-to-avoid-a-page-break-before-start-of-bibliography
\begingroup
\let\clearpage\relax
\bibliographystyle{plain}
\bibliography{References}
\endgroup

%\bibliography{References}
\begingroup
\let\clearpage\relax
\listoffigures
\endgroup

\begingroup
\let\clearpage\relax
\listoftables
\endgroup

\begingroup
\let\clearpage\relax
\lstlistoflistings
\endgroup

% Optional ein Sachwortverzeichnis
%\newpage
%\printglossary[type=\acronymtype]
%\addcontentsline{toc}{section}{Stichwortverzeichnis}
%\clearpage
%\thispagestyle{scrheadings}
%\printindex
%\printglossary
\newpage
\section*{EIDESSTATTLICHE ERKLÄRUNG}
\vspace{6cm}
Hiermit erkläre ich ehrenwörtlich, dass ich die vorliegende Arbeit selbstständig angefertigt, andere als die angegebenen Quellen und Hilfsmittel nicht benutzt und die den benutzten Quellen wörtlich oder inhaltlich entnommenen Stellen als solche kenntlich gemacht habe.

Ich erkläre außerdem, dass die vorliegende Arbeit bei keiner anderen Institution (Fachhochschule, Universität, Pädagogische Hochschule oder vergleichbare Bildungseinrichtung) zur Erlangung eines akademischen Grades eingereicht wurde.
\vspace{3cm}
\\
---------------------------------------\hspace*{4cm}--------------------------
\\\hspace*{1.5cm}Ort, Datum\hspace*{6.15cm}Unterschrift

\end{document}